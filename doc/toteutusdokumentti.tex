\documentclass[12pt]{article}

\usepackage{graphicx}
\usepackage[finnish]{babel}
\usepackage[utf8]{inputenc}
\usepackage{listings}

\begin{document}

\begin{titlepage}
\title{Toteutusdokumentti}
\author{Janne Ronkonen}
\maketitle
\end{titlepage}

\tableofcontents

\section{Johdanto}


\subsection{Rajaukset}

Ei käyttäjiä

Ei reseptin jakoa osiin (esim. kastike, perunat)

\section{Ohjelmiston yleisrakenne}

\section{Järjestelmän komponentit}

\section{Asennustiedot}


Järjestelmä vaatii toimiakseen Common Lisp -toteutuksen sekä PostgreSQL -tietokantapalvelimen. 


Lisäksi tarvitaan seuraavat Common Lisp-kirjastot:

- Hunchentoot: web-serveri (suositellaan ajettavaksi esim. Apachen takana proxylla)
- Postmodern: tietokanta-ajuri Postgresille
- LML2: HTML:n generointia varten
- CL-PPCRE: Perl Compatible Regular Expressions

Kirjastojen asennus on helpointa Quicklisp -kirjastojen asennusohjelman avulla.

Tiedosto build.lisp sisältää skriptin joka kääntää SBCL -Lisp-toteutusta käyttäen ohjelman
yhdeksi binääritiedostoksi jonka voi sitten ajaa komentoriviltä. Käännetty ohjelma avaa
käynnistyessään Lisp-komentorivin jonne kirjoittamalla `(tsoha:start [porttinumero])´ saa ohjelman
käynnistymään halutussa portissa. Jos porttinumeroa ei anneta, käytetään oletusta joka on 8080.
Ohjelman voi sammuttaa kirjoittamalla komentoriville `(tsoha:stop)´ (tai tappamalla ko. prosessin).





\section{Käyttöohjeet}


\appendix
\section{Tehtäväkuvaus}

\appendix
\section{Suunnitteludokumentti???}

\appendix{Create table -lauseet jos ei suunnitteludokumentissa}

\appendix{Ohjelmakoodit}


\end{document}
